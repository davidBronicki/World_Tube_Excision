\documentclass[12pt]{article}
%\usepackage[utf9]{inputenc}
%\documentclass[aps,onecolumn,12pt]{revtex4-1}
%\documentclass[aps,10pt]{revtex4-1}
\usepackage{amsmath}
\usepackage{amssymb}
\usepackage{amsfonts}
\usepackage{amsthm}
\usepackage{float}
\usepackage{color}
\usepackage{graphicx}
\usepackage{gensymb}
\usepackage{caption}
\usepackage{subcaption}
\usepackage{wrapfig}
\usepackage{braket}
\usepackage{verbatim}
\usepackage{mathtools}
\usepackage{soul}
\usepackage[margin=0.8in]{geometry}

\usepackage[dvipsnames]{xcolor}

\usepackage{fancyvrb}

\renewcommand{\bf}[1]{\boldsymbol{#1}}

\begin{document}
\title{Notes on World Tube Excision: Mutlipole Expansion in Electrodynamics}
\author{David Bronicki}

\maketitle

All calculations use $c=\varepsilon_0=1$.

\section{Static Field}

\subsection{Without external charges}

For three dimensions, we can copy directly from Jackson
(with a shift of $2\ell+1$ from inside the expansion
to in the definition of the coefficient):

\begin{align}
	\Phi&=\sum_{\ell m}q_{\ell m}
	\frac{Y_{\ell m}(\theta, \phi)}{r^{\ell+1}}\\
	q_{\ell m}&=
	\int \frac{1}{2\ell+1} Y_{\ell m}^*(\theta',\phi'){r'}^\ell\rho(\bf x')d^3x'.
\end{align}

For two dimensions (as I would like to start with in Python),
we start with the Green's function in polar coordinates:

\begin{align}
	\nabla^2 G(\bf x, \bf x')&=\delta(\bf x - \bf x')\\
	G(\bf x, \bf x')&=\ln(r_>)+
	\sum_{m\neq 0}e^{im(\phi-\phi')}
	\frac{-1}{2|m|}\left(\frac{r_<}{r_>}\right)^{|m|}\\
	&=\ln(r_>)-
	\sum_{m>0}\frac{1}{m}\cos(m(\phi-\phi'))
	\left(\frac{r_<}{r_>}\right)^m\\
	&=\ln(r_>)-
	\sum_{m>0}\frac{1}{m}\left(\frac{r_<}{r_>}\right)^m
	\left[\sin(m\phi)\sin(m\phi')+\cos(m\phi)\cos(m\phi')\right],
\end{align}
where $r_<=\min(r,r')$ and $r_>=\max(r,r')$.
Following Jackson's derivation, we proceed to
integrate over all charge and assume the point
of observation ($\bf x$) is further from the
origin than all charges in the configuration.
That is, we set $r_<=r'$ and $r_>=r$. This gives
\begin{align}
	\Phi&=-q_0 \ln(r)+\sum_{m\neq 0}q_m e^{im\phi}\frac{1}{r^{|m|}}\\
	q_0&=\int\rho(\bf x') d^2x'\\
	q_{m\neq 0}&=\int \frac{1}{2|m|}e^{-im\phi'}{r'}^{|m|}\rho(r',\phi')d^2x'
\end{align}

\subsection{With unknown external charges/field}
To find boundary constraints which handle arbitrary external
charges (and hopefully fields), I added in arbitrary coefficients
and created constraints on the potential and it's $r$ derivative at
some boundary $r=R$.
In two dimensions:
\begin{align}
	\Phi\big|_R&=\left[-q_0\ln(R)+Q_0\right]
	+\sum_{m\neq 0}\left[
		Q_m R^{|m|}+\frac{q_m}{R^{|m|}}\right]e^{im\phi}\\
	\left.\frac{\partial \phi}{\partial r}\right|_R&=
	\frac{q_0}{R}
	+\frac{1}{R}\sum_{m\neq 0}|m|\left[
		Q_m R^{|m|}-\frac{q_m}{R^{|m|}}\right]e^{im\phi}.
	\intertext{We now define coefficients of the expansion on this boundary:}
	c_m&=\int \Phi(R,\phi')e^{-im\phi'}d\phi'\\
	\Phi\big|_R&=\sum_m c_m e^{im\phi}
	\intertext{and similar for the normal derivative}
	c_m'&=\int \left.\frac{\partial\Phi}{\partial r}\right|_{R,\phi'}
	e^{-im\phi'}d\phi'\\
	\left.\frac{\partial\Phi}{\partial r}\right|_R&=\sum_m c_m' e^{im\phi}
	\intertext{and get the constraint}
	\Aboxed{R^{|m|}\big[|m| c_m-c_m' R\big]&=
	\int \rho(\bf x')e^{-im\phi'}{r'}^{|m|}d^2x'.}
\end{align}
(The last integral is over only internal charges.)

Similarly, in three dimensions,
\begin{align}
	\Phi\big|_R&=\sum_{\ell m}\left[
		q_{\ell m}\frac{1}{R^{\ell+1}}
		+Q_{\ell m}R^\ell\right]
	Y_{\ell m}(\theta, \phi)\\
	\left.\frac{\partial\Phi}{\partial r}\right|_R&=
	\frac{1}{R}\sum_{\ell m}\left[
		-q_{\ell m}\frac{\ell+1}{R^{\ell+1}}
		+\ell Q_{\ell m}R^\ell\right]
	Y_{\ell m}(\theta, \phi),
	\intertext{with similarly defined coefficients for the boundary:}
	c_{\ell m}&=\int \Phi(R,\theta',\phi')Y_{\ell m}^*(\theta',\phi')
	\sin(\theta')d\theta'd\phi'\\
	\Phi\big|_R&=\sum_{\ell m} c_{\ell m} Y_{\ell m}(\theta, \phi)\\
	c_{\ell m}'&=\int \left.\frac{\partial\Phi}{\partial r}\right|_{R,\theta',\phi'}
	Y_{\ell m}^*(\theta',\phi')\sin(\theta')d\theta'd\phi'\\
	\left.\frac{\partial\Phi}{\partial r}\right|_R&=\sum_{\ell m}
	c_{\ell m}' Y_{\ell m}(\theta, \phi)
	\intertext{yield beautifully similar constraints:}
	\Aboxed{R^{\ell+1}\big[\ell c_{\ell m}-R c_{\ell m}'\big]&=
	\int Y_{\ell m}^*(\theta',\phi'){r'}^\ell
	\rho(\bf x')d^3x',}
\end{align}
again with the last integral over only internal charges.

\end{document}
